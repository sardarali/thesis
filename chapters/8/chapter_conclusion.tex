\startchapter{Conclusions and Future Work}
\label{chap:conclusion}
\section{Conclusions}
Power quality plays an important role in the reliability of power grids. To improve the reliability of the power grid networks, power quality (PQ) measurement devices are being deployed to monitor the power quality on underlying links. Since power quality meters are expensive devices, it is financially infeasible to install them on every link between devices in the power grid network. In the first part of this thesis, we studied the problem of estimating the power quality on unmonitored network segment based on power quality readings from the monitored segments. We modeled the power grid network as a data network to leverage the existing network and data estimation techniques to solve the problem. To solve the power quality estimation on unmonitored links,  we proposed a maximum entropy (MaxEnt) based model that accurately estimates power quality transition functions. Through experimental evaluations, we showed that our MaxEnt based solution is much faster and accurate than the existing expectation maximization (EM) based solution.

Next, we investigated the problem of intelligently placing measurement devices on suitable power links to reduce the uncertainty of PQ estimation on unmonitored power links. As a first step to tackle the meter placement challenge, we represented the latent feature of a device as a transition function which is usually estimated through physical modeling or through the assessment of historical power monitoring data. Using a real PQ dataset, we showed that historical data can be used to capture the latent features of a device. After learning the device latent features, we then proposed intelligent meter placement algorithms for identifying network segments suitable for power meter placement. The two algorithms that solve the meter placement problem are: 1)  an intelligent conditional entropy (CE)-based algorithm; and 2) a Bayesian network (BN)-based approach. The two algorithms were evaluated in various simulation networks including the IEEE 13-node distribution test feeder. Results suggested that the CE-based MinEntropy approach is much faster. Further, the proposed solutions significantly reduced the uncertainty of PQ values on unmonitored power links.

Finally, using the power quality readings/events from monitored segments, we identified statistical features and proposed a model that accurately identify a potential malfunction device in the power network. The proposed solution was evaluated on various simulated power networks including the IEEE Test Feeder network. Our evaluations confirmed the accuracy of the proposed solution. The proposed malfunction segment detection solution will help in the reliability improvement of power networks.

\section{Future Work}
\subsection{Scaling the MaxEnt-based PQ Estimation}
Our proposed MaxEnt-based model opens a new scope of methods to quantitatively measure and solve the reliability problems in smart grid. The proposed solution may not converge efficiently for sub-nets of larger sizes. The scalability problem arises as the number of unknown variables increases exponentially with the increase in the number of components in a subnet. The idea for the extended work is to divide the larger subnets in logical components where each logical component will represent several physical components. Instead of computing individual transfer functions for each physical device, transfer functions of the logical components could be estimated first. In the second round, transfer functions of individual physical components could be estimated from the logical transfer functions. Note that this is a high level guideline which only serves as a starting point towards the final solution. The final solution could be based on a comprehensive mathematical model. Finally, for the evaluation of such a system, various larger IEEE standard test networks could be used.


\subsection{Extending the Meter Placement Solution to Larger Networks that Contain Loops}
Our meter placement solution could be used for meter placement in several standard power networks including the 13-Node IEEE Test Feeder. As a future work, one could look into the possibility of extending our work to cover the networks that contain loops to relax our tree network assumption.

\subsection{Device Level Misbehavior Detection}
In this thesis, we have proposed statistical measures that help in detecting a potential malfunction device in the power network. If the device is not directly monitored, e.g., it is somewhere in the middle of a monitored segment, we can only detect the segment and not the device explicitly. As a future work, the solution could be improved to detect the malfunction device explicitly that is not directly monitored. Further, it could also be investigated if the accumulative suffer time of the device (using the standard measures defined in the CBEMA curve) could be used to make the device maintenance/replacement recommendations.