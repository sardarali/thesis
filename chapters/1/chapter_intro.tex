\startfirstchapter{Introduction}
\label{chapter:introduction}
\section{Why Power Quality Monitoring?}
Electrical power networks are one of the critical infrastructures of our society. Due to our high dependence on electricity, the issue of reliability in electric networks has become a core research interest in the area of smart grid~\cite{Moslehim10}. Reliability evaluation of power grid, however, is challenging due to the existence of multiple electric utilities and the potential of cascading failures of power distribution systems~\cite{Albert04, chen2010cascading, lopez2011challenges, parandehgheibi2014mitigating}. One of the most influential factors impacting the reliability and energy saving of power networks is the power quality delivered to, and experienced by, critical electric equipment. Poor power quality, such as voltage sags/swells, harmonics, fast impulses etc, may lead to power outage and service interruptions. Service unavailability caused by power losses is a serious problem for many companies and organizations, e.g., it may result in a significant revenue loss for Internet service providers or even loss of lives in hospitals. To improve the reliability of power networks, organizations and large companies (e.g., Google data centers) adopt smart microgrid, and closely monitor the power quality in different segments of the microgrid. Hence, the monitoring of power quality is a crucial component of assessing and maintaining reliability in power grids.

Monitoring power quality, however, is not an easy task. Since the power measurement devices~\cite{fluke_meter, schneider_meter} are expensive, it is financially impractical to monitor every segment of a power network. The overhead of interconnecting these power meters and developing the power management system further increases the cost. In addition, in many cases direct monitoring of power quality is difficult, e.g., it is hard to install smart meters after power lines were sealed in hard-to-reach areas in a building. We identify various research challenges in power quality monitoring.

\section{Open Challenges}
In order to effectively monitor the power quality in the smart grid, this work is intended to tackle the following challenges:

\begin{enumerate}
\item Based on a limited small number of monitored points in a power network, how can we effectively estimate the power quality of other unmonitored segments of the network?

\item Given a fixed number of available power meters, which grid segments should be selected for monitoring such that power quality can be inferred as accurately as possible in the remaining unmonitored segments of the network? A relevant research problem is to design a mechanism that calculates the optimal number of meters required to achieve an acceptable level of network reliability.

\item Based on readings from our meters installed in the power grid, how can we accurately detect a potential malfunction device?

\end{enumerate}

As the first step to tackle the above challenges, the probabilistic calculation of power quality values on unmonitored links requires the behavior (latent feature) of each device to be known. We represent the latent feature of a device as a transfer function which is usually estimated through physical modeling or through the assessment of historical power monitoring data. Using a real power quality dataset, we show that historical data can be used to capture the latent features of a device. Our device latent feature model is presented in Chapter~\ref{chap:latentF}.

With devices' latent features captured, we in the second step introduce a network model of the smart microgid as a data-driven network, in analogy, where we represent the electrical components as network nodes, power links as data links, flow of power as data flow on the links, and the power flowing through links as numeric data. The power quality estimation problem can then be modeled as an optimization problem of missing data estimation in a data network. This problem transformation significantly simplifies the complexity of the power grid network; it also gives us the opportunity to use the well-investigated network monitoring and data estimation algorithms to solve the network quality monitoring in power grids. Chapter~\ref{chap:networkModel} presents the proposed network model.

Finally, using our network model, we propose various algorithms to tackle the three challenges we identified in this section. In the next section, we detail the identified challenges and summarize the solutions we propose to solve each challenge.

\section{Proposed Solutions}
Power quality meters play an important role in the reliability evaluations of the power grids. In this thesis, we identify three research problems related to smart meter placement, power quality estimation on unmonitored segments, and detection of a malfunction device in the power grid. Summary of the our proposed solutions for the three research challenges is as follows:

\subsection{Power Quality Estimation using Maximum-Entropy Approach}
The reliability evaluation of enterprise-level power microgrid seems to be much simpler compared to the large-scale power grid which is notoriously difficult due to the existence of the multiple electric utilities and the cascading failures of power distribution systems~\cite{Albert04}. Nevertheless, to tackle the practical challenges, the power quality and operational status of electric devices in the micogrid must be monitored and recorded. On the other hand, due to financial and other practical issues, not all devices in the network can be monitored. We need to tackle the following challenge: \textit{Based on a limited small number of monitored points in a power network, how can we effectively estimate the power quality of other unmonitored segments of the network?}

We propose to use a Maximum-Entropy (MaxEnt)~\cite{maxent} approach to power quality estimation. The basic idea of MaxEnt is that out of all probability distributions consistent with a given set of constraints, we should choose the one that has the maximum uncertainty to be the estimated power quality values. Intuitively, the principle of MaxEnt implies that we should make use of all the information that is given and avoid making (biased) assumptions about information that is not available.

The MaxEnt approach is built on the top of our network model which gives us the opportunity to use existing data estimation techniques used in the data networks. The problem of estimating power quality is modeled in such a way where we effectively get the benefit of MaxEnt approach to correctly estimate the power quality values at unmonitored links. We solve the formulated MaxEnt problem and validate its effectiveness and efficiency with a simulated microgrid system. Compared to other existing methods such as Monte Carlo Expectation Maximization (MCEM), the MaxEnt based approach is much faster. The proposed MaxEnt approach is presented in Chapter~\ref{chap:PQEstimation} and has been published in~\cite{ali2013maximum}.

\subsection{Intelligent Meter Placement using Bayesian Network, and Conditional Entropy-based Approaches}
Power quality meters are being deployed to monitor the power quality in the power grid network. Power quality meters are expensive devices~\cite{fluke_meter, schneider_meter} and it is impractical to monitor the power quality at every segment in the power grid network. Instead, power quality in unmetered grid locations must be inferred given data obtained from the measured locations. The research question arise is \textit{where to place the meters in the power grid network?}
	
We propose an iterative approach for identifying network segments suitable for power meter placement. During each iteration of the algorithm we identify in a greedy manner the network segment that suffers from the most unpredictable power quality given the meters deployed so far. We then deploy the next power meter at that location.

A relevant challenge here is to identify the optimal number of meters to reduce the uncertainty and hence the overall reliability of the network to an acceptable level. Formally, we tackle the problem of \textit{how to design a mechanism that calculates the optimal number of meters required to reduce the uncertainty of power quality in the power grid to an acceptable level?} We propose to model the above issue as an optimization problem to minimize the number of meters while maintaining the desired level of network reliability.

For above two problems, the detailed problem definitions, the proposed solutions, and results from an experimental study are presented in details in Chapter~\ref{chap:meterPlacement}. The Bayesian network-based solution has been published in~\cite{ali2013intelligent} while the conditional entropy-based solution is accepted for publication in~\cite{alimachine}. A U.S. patent~\cite{marinakis2014systems} covering the proposed meter placement methods has been granted.

\subsection{Detecting a Malfunction Device using Our Prediction Model}
The main objective of this work is to reduce the cost of power quality monitoring while ensuring the reliability of the power grid network. The two research problems discussed above address how to accurately estimate the state of the network. This information could be used to avoid device failures. We need to propose a model that could accurately detect any significant change in the normal behavior of a device. By doing so, we would be able to raise an alarm and make recommendations for the device maintenance or possible replacement before the device significantly compromise the reliability (in terms of power quality) of the power grid. The research question is: \textit{how to detect a potential malfunction device in the power network based on available PQ readings}.

To address the challenge, using the power quality readings from the monitored links, we propose statistical measures that accurately detect a potential malfunction device in the power network. Our proposed solution and the simulation results of its accuracy evaluations are presented in Chapter~\ref{chap:predictionModel}.

\section{Contributions}
The proposed thesis work investigate various algorithms to tackle our three research challenges in the area of power quality monitoring in power grid. As the first step to tackle the above challenges, we represent a device latent feature model used to capture the behavior of the devices in the power grid. With devices' latent features captured, we in the second step introduce a network model of the smart microgid as a data-driven network. This problem transformation significantly simplifies the complexity of the power grid network; it also give us the opportunity to use the well-investigated network monitoring and data estimation algorithms to solve the network quality monitoring in power grids.

Our latent feature and network models are detailed in separate chapters in this thesis. Using the network model, we propose various algorithms to tackle the three challenges and make the following three major contributions:

\begin{enumerate}
\item \textbf{Power Quality Estimation:} A Maximum-Entropy (MaxEnt) approach to power quality estimation. The proposed solution is presented in Chapter~\ref{chap:PQEstimation}.
\item \textbf{Intelligent Meter Placement:} An intelligent entropy-based algorithm and a Bayesian network based approach to solve the meter placement problem. The proposed meter placement algorithms and their detailed evaluations are presented in Chapter~\ref{chap:meterPlacement}.
\item \textbf{Malfunction Device Detection:} Based on statistical measures, we propose a prediction model to detect a potential malfunction device in the network. Using the inferred and actual PQ values by meters we placed using our intelligent meter placement algorithms. The proposed model with its simulation results is presented in Chapter~\ref{chap:predictionModel}.
\end{enumerate}

\section{Thesis Outline}
The rest of the thesis is organized as follows. Chapter~\ref{chap:review} provides review on power quality in smart grid and discusses the available literature related to our proposed work. The proposed device latent feature model is presented and evaluated on a real dataset in Chapter~\ref{chap:latentF}. Our network model of the power grid is presented in Chapter~\ref{chap:networkModel}. Based on the proposed network model, we build various algorithms to address the identified research issues. In Chapter~\ref{chap:meterPlacement}, we propose algorithms that intelligently place the power meters on high information locations in the power grid. The research issue of estimating power quality values on unmonitored links is investigated in Chapter~\ref{chap:PQEstimation}. In Chapter~\ref{chap:predictionModel}, based on the known power quality values from our proposing algorithms, we present a prediction model that detects a potential malfunction device. The thesis in concluded and the possible future extensions are discussed in Chapter~\ref{chap:conclusion}.