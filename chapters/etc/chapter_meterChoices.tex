\startchapter{Meter Selection Criteria}
\label{chap:meterChoices}
The rapid increase in power energy needs catalyzed the massive growth of electric networks. In order to meet the growing energy demand, electric industry seek to take advantage of novel approaches. Further, the need to merge the renewable distributed power generations with the legacy grids is driving the electric grid to a new grid paradigm -- smart grid. According to a recent survey on smart grid, the research is focusing in the area is  focusing on three systems in -- the infrastructure system, the management system, and the protection system~\cite{fang2011smart}. The smart grid infrastructure system consists of energy, information, and communication infrastructures where the smart metering is one of the key components.

Since the electricity grids consist of networks of varying physical characteristics, the capabilities of power quality meters varies accordingly. Second, there are many causes of power quality problems including voltage sags/swells, fast/sub-cycle impulses, harmonics, and high frequency noise etc, which will influence the choice of meters to be deployed. Another factor could be the frequency and potential effect of these causes as some of the causes occur frequently while the others occur rarely. Finally, depending on the users' requirements and financial budget, the capabilities of the power meters varies.

We plan to conduct a detailed research study to identify the required capabilities of a power meter based on: 1) user types; 2) financial budget; and 3) the environmental factors. Currently, many measurement devices use the level of voltage on electric lines as the measurement parameter to classify power quality. In additional to using the voltage level as a measure, we propose to consider physical power characteristics like level of current as well.

We plan to conduct a detailed study on the issue discussed above. Specifically we plan to: 1) study the causes and effects of PQ problems; 2) physical power characteristics on segments of different kind of networks; 3) the influence of environmental factors on power d; and 4) other driving factors like financial budget of the user etc. Finally, based on our findings, we will give our recommendations about what kind of power quality meter need to be deployed in different scenarios.