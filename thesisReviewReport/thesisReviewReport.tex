\documentclass[journal]{IEEEtran}
\usepackage{setspace}
\usepackage{color}
\usepackage{changepage}
\usepackage{hyperref}
\usepackage{multirow}
\usepackage{amsmath}
\usepackage{amssymb}
\usepackage{caption}
\usepackage{scalerel,stackengine}
\stackMath
\newcommand\widecheck[1]{%
\savestack{\tmpbox}{\stretchto{%
  \scaleto{%
    \scalerel*[\widthof{\ensuremath{#1}}]{\kern-.6pt\bigwedge\kern-.6pt}%
    {\rule[-\textheight/2]{1ex}{\textheight}}%WIDTH-LIMITED BIG WEDGE
  }{\textheight}% 
}{0.5ex}}%
\stackon[1pt]{#1}{\scalebox{-1}{\tmpbox}}%
}

\usepackage{enumitem}
\onecolumn
\newcommand{\nop}[1]{}

\newcommand{\specialcell}[2][c]{%
  \begin{tabular}[#1]{@{}c@{}}#2\end{tabular}}
\makeatletter

\begin{document}
\title{Thesis Revision Report}
\maketitle

 \Large
\noindent \textbf{\underline{Response to Dr. Hao Liang:}}
 \large

\vspace{10pt}
\textbf{Comment \#1:}
\begin{adjustwidth}{2.5em}{0pt}
\singlespacing \vspace{-10pt}
\textcolor{red}{Title: Smart Power Micro-grid (Enterprise level grid). Utility level micro-grid (more issues, voltage unbalance, disturbance, so on).}
\end{adjustwidth}

\vspace{5pt}
\textbf{Response:}
\begin{adjustwidth}{2.5em}{0pt}
I have changed the title to:

\vspace{5pt}
\noindent\textcolor{blue}{An Analytical Framework for Power Quality Monitoring in Enterprise Level Power Grid}
\end{adjustwidth}
 
 
 
\vspace{30pt}
\textbf{Comment \#2:}
\begin{adjustwidth}{2.5em}{0pt}
\singlespacing \vspace{-10pt}
\textcolor{red}{Chapter 1: Add related work, before contributions}
\end{adjustwidth}

\vspace{5pt}
\textbf{Response:}
\begin{adjustwidth}{2.5em}{0pt}
I have updated the Introduction chapter by adding related work (before contributions). The new section (Section 1.4) added is as follows:
\end{adjustwidth}

\vspace{5pt}
\begin{adjustwidth}{2.5em}{0pt}
\noindent\textcolor{blue}{{\Large Existing Solutions to Power Quality Monitoring} \vspace{0.4em} \\
The existing solutions are divided into two categories: 1) meter placement; and 2) power quality estimation. The meter placement problem is related to optimal sensor/PMU placement and there is a great body of work on sensor and PMU placement [27--47]. These solutions are targeting specific applications/areas in the power systems (detailed in Section 2.3.4). Nevertheless, we have not seen any work on studying optimal meter placement problem in the context of network-wide power quality estimation. Further, there are three major differences between the existing PMU placement algorithms and our meter placement algorithms.}\end{adjustwidth}

\vspace{-1em}
\begin{adjustwidth}{3.5em}{0pt}
\noindent\textcolor{blue}{
\begin{enumerate}
\item We focus on distribution networks at the enterprise level (e.g., a university
campus).
\item Our method is data driven and is based on statistical machine learning method.
\item The existing PMU placement algorithms address the problem of estimating network states and do not consider power quality estimation explicitly. Further, each PMU solution targets a specific problem in the power network (detailed in Section 2.3.4) and hence the objective function and problem parameters (e.g., phase angle) are different. In other words, these solutions are mathematically different from the meter placement solutions we proposed in this thesis.
\end{enumerate}
}
\end{adjustwidth}

\begin{adjustwidth}{2.5em}{0pt}
\noindent\textcolor{blue}{The power quality estimation problem was addressed in [72] using the expectation maximization (EM) algorithm. Compared to the EM based algorithm, our proposed MaxEnt solution significantly improves the running time while maintaining the accuracy of the power quality values estimated. The running time is particularly important when the network size becomes larger and the power quality needs to be estimated in real-time.}
\end{adjustwidth}

\vspace{30pt}
\textbf{Comment \#3:}
\begin{adjustwidth}{2.5em}{0pt}
\singlespacing \vspace{-10pt}
\textcolor{red}{Power quality monitoring: Regular order: placement --\textgreater monitoring --\textgreater estimation. Why do you estimate first? Any reason for the change of order? Add some discussion over there.}
\end{adjustwidth}

\vspace{10pt}
\textbf{Response:}
\begin{adjustwidth}{2.5em}{0pt}
I have reordered the chapters. Meter placement is now Chapter 5 while Estimation is Chapter 6.
\end{adjustwidth}



\vspace{20pt}
\textbf{Comment \#4:}
\begin{adjustwidth}{2.5em}{0pt}
\singlespacing \vspace{-10pt}
\textcolor{red}{Page 12: Table 2.2. Swell (double check all values. Swell values normally are larger than 1).}
\end{adjustwidth}

\vspace{10pt}
\textbf{Response:}
\begin{adjustwidth}{2.5em}{0pt}
Thanks for identifying this mistake. I double checked all the values in the table. The swell starting values are also changed from $0.1$ to $1.1$
\end{adjustwidth}



\vspace{30pt}
\textbf{Comment \#5:}
\begin{adjustwidth}{2.5em}{0pt}
\singlespacing \vspace{-10pt}
\textcolor{red}{Page 17: Item 3. “The existing PMU placement algorithms address the problem of estimating network states and do not consider power quality estimation explicitly.” More explanation is required. What is the main difference between your problem and existing PMU Placement problem mathematically?}
\end{adjustwidth}

\vspace{10pt}
\textbf{Response:}
\begin{adjustwidth}{2.5em}{0pt}
I have added a new section (Section 1.4) in Chapter 1 to explain the difference between PMU and our meter placement solutions. The new section added is as follows.

\end{adjustwidth}

\vspace{5pt}
\begin{adjustwidth}{2.5em}{0pt}
\noindent\textcolor{blue}{{\Large Existing Solutions to Power Quality Monitoring} \vspace{0.4em} \\
The existing solutions are divided into two categories: 1) meter placement; and 2) power quality estimation. The meter placement problem is related to optimal sensor/PMU placement and there is a great body of work on sensor and PMU placement [27--47]. These solutions are targeting specific applications/areas in the power systems (detailed in Section 2.3.4). Nevertheless, we have not seen any work on studying optimal meter placement problem in the context of network-wide power quality estimation. Further, there are three major differences between the existing PMU placement algorithms and our meter placement algorithms.}\end{adjustwidth}

\vspace{-1em}
\begin{adjustwidth}{3.5em}{0pt}
\noindent\textcolor{blue}{
\begin{enumerate}
\item We focus on distribution networks at the enterprise level (e.g., a university
campus).
\item Our method is data driven and is based on statistical machine learning method.
\item The existing PMU placement algorithms address the problem of estimating network states and do not consider power quality estimation explicitly. Further, each PMU solution targets a specific problem in the power network (detailed in Section 2.3.4) and hence the objective function and problem parameters (e.g., phase angle) are different. In other words, these solutions are mathematically different from the meter placement solutions we proposed in this thesis.
\end{enumerate}
}
\end{adjustwidth}

\begin{adjustwidth}{2.5em}{0pt}
\noindent\textcolor{blue}{The power quality estimation problem was addressed in [72] using the expectation maximization (EM) algorithm. Compared to the EM based algorithm, our proposed MaxEnt solution significantly improves the running time while maintaining the accuracy of the power quality values estimated. The running time is particularly important when the network size becomes larger and the power quality needs to be estimated in real-time.}
\end{adjustwidth}



\vspace{30pt}
\textbf{Comment \#6:}
\begin{adjustwidth}{2.5em}{0pt}
\singlespacing \vspace{-10pt}
\textcolor{red}{Page 22, Figure 3.1: Circle represents the power quality meters. What types of power quality meters are used here? How about devices that only have one link to the smart grid (e.g. Capacitor bank)? It is better to change the title of the thesis and narrow down to Enterprise level power grid.}
\end{adjustwidth}

\vspace{5pt}
\textbf{Response:}
\begin{adjustwidth}{2.5em}{0pt}
I have changed the title to:

\vspace{5pt}
\noindent\textcolor{blue}{An Analytical Framework for Power Quality Monitoring in Enterprise Level Power Grid}
\end{adjustwidth}


\vspace{30pt}
\textbf{Comment \#7:}
\begin{adjustwidth}{2.5em}{0pt}
\singlespacing \vspace{-10pt}
\textcolor{red}{Data collection: details of the data. How large is the size, voltage level?}
\end{adjustwidth}

\vspace{10pt}
\textbf{Response:}
\begin{adjustwidth}{2.5em}{0pt}
I have provided details about the network and voltage levels as follows.

\vspace{10pt}
\noindent\textcolor{blue}{Our power quality dataset was collected at an enterprise power network for a period of four years. For privacy and security reasons, the physical network structure/diagram is omitted. Instead, we represent the topology/positions of the installed power quality meters via a graph network as shown in Fig. 3.2. There are a total of $10$ power quality meters (numbered from $m_1$ to $m_{10}$) installed. Each meter reported the power quality events (sag/swell, transient, etc.) to the data collection server via Ethernet network. It is important to mention that we currently do not consider power transmission network, which is large-scale and may involve multiple utilities across a country, but only focus on power distribution network at the enterprise-level, e.g., university campus. Hence, we collect the power quality dataset at an enterprise network located at the distribution level. The network is using a standard three-phase distribution system. Devices of varying loads are using this network, including electric vehicles and large motors. Three-phase transformers with four-wire output are used for $120$ volt service. Table 3.1 shows the number of events reported by each power quality meter while the positions of the meters are shown in Fig. 3.2.}
\end{adjustwidth}


\vspace{30pt}
\textbf{Comment \#8:}
\begin{adjustwidth}{2.5em}{0pt}
\singlespacing \vspace{-10pt}
\textcolor{red}{14 power quality events. Explain the meaning of Power quality class (Table 3.2).}
\end{adjustwidth}

\vspace{10pt}
\textbf{Response:}
\begin{adjustwidth}{2.5em}{0pt}
I have added a new table that list the power quality classes (classification) defined be the IEEE std. 1159.
\end{adjustwidth}

\begin{table}[!h]
\center \color{blue}
\caption*{\color{blue}Table 3.4: Power Quality Event Classification Defined by IEEE Standard 1159-2009 [2].}
\centering \renewcommand*{\arraystretch}{1.3} 
\begin{tabular}{|c|l|c|c|c|c|}
\hline \multirow{2}{*}{\textbf{PQ Class}} & \multirow{2}{*}{\textbf{Event Type}} & \multicolumn{2}{c}{\textbf{Voltage (\% nominal)}} & \multicolumn{2}{|c|}{\textbf{Duration (seconds)}} \\
\cline{3-6}   &  & \textbf{Min} & \textbf{Max} & \textbf{Min} & \textbf{Max} \\ 
\hline  $c_1$ & Microsecond Transient  & 0   & unlimited & 0           & 0.001    \\
$c_2$ & Millisecond Transient  & 0   & unlimited & $>$0.001    & 0.008333 \\
$c_3$ & Instantaneous Sag      & 10  & 90        & $>$0.008333 & 0.5      \\
$c_4$ & Instantaneous Swell    & 110 & unlimited & $>$0.008333 & 0.5      \\
$c_5$ & Momentary Interruption & 0   & $<$10     & $>$0.008333 & 3        \\
$c_6$ & Momentary Sag          & 10  & 90        & $>$0.5      & 3        \\
$c_7$ & Momentary Swell        & 114 & unlimited & $>$0.5      & 3        \\
$c_8$ & Temporary Interruption & 0   & $<$10     & $>$3        & 60       \\
$c_9$ & Temporary Sag          & 10  & 90        & $>$3        & 60       \\
$c_{10}$ & Temporary Swell        & 110 & unlimited & $>$3        & 60       \\
$c_{11}$ & Sustained Interruption & 0   & $<$10     & $>$60       & unlimited\\
$c_{12}$ & Undervoltages          & 10  & 90        & $>$60       & unlimited\\
$c_{13}$ & Overvoltages           & 110 & unlimited & $>$60       & unlimited\\
\hline $c_{14}$ & Nominal                & \multicolumn{4}{c|}{Anything not covered above} \\ \hline
\end{tabular}
\end{table}


\vspace{30pt}
\textbf{Comment \#9:}
\begin{adjustwidth}{2.5em}{0pt}
\singlespacing \vspace{-10pt}
\textcolor{red}{Table 3.6: What is the device of D8? Why the output of C14 is high?  To help readers understand, detail of D8 needs to give.}
\end{adjustwidth}

\vspace{10pt}
\textbf{Response:}
\begin{adjustwidth}{2.5em}{0pt}
I agree with the fact that the frequency of the PQ class $c_{14}$ for all meters is reported as 0 (in Table 3.2) while the same for a single meter (device $d_8$ in Table 3.6) is reported as $2122$. The reason is as follows.

\vspace{5pt}\noindent The power quality classes from $c_1$ to $c_{13}$ represent the power disturbances (described in Table 3.4) while $c_{14}$ represents the nominal power quality (a good/normal power quality). The power quality meters in our data collection network were configured to report only bad power quality events and hence the frequency of PQ class $c_{14}$) is 0 in table 3.2. We also noticed that, in some cases, there are bad power quality events reported at some links while nothing reported by other meters at that time instance. This happens when a device, for instance a UPS, maps a bad quality to good quality. In such cases, we assume a nominal PQ value (PQ class $c_{14}$) at the monitored but unreported points.

\vspace{5pt}\noindent The same explanation is now added in Chapter 3. The revised explanation from the thesis is as follows.

\vspace{5pt}
\noindent\textcolor{blue}{Using IEEE Standard 1159 [2], we classify the power quality events based on the fluctuation of the voltage for a predefined period. There are $14$ different power quality classes defined in the standard, denoted from $c_1$ to $c_{14}$, respectively. Table 3.5 shows samples of the events we classify using the IEEE standard where the power quality class is shown in the last column of the table. The frequency of events belonging to the IEEE power quality class ($c_1$ to $c_{14}$) is shown in Table 3.2. Description of the IEEE power quality classes is provided in Table 3.4.}

\vspace{5pt}
\noindent\textcolor{blue}{The power quality meters in our data collection network were configured to report only bad power quality events. Therefore, the frequency of the nominal voltage events (PQ class $c_{14}$) in Table 3.2 is 0. We noticed that, in some cases, there are bad power quality events reported at some links while nothing reported by other meters at that time instance. This happens when a device, for instance a UPS, maps a bad quality to good quality. In such cases, we assume a nominal PQ value (PQ class $c_{14}$) at the monitored but unreported points.
}
\end{adjustwidth}

\vspace{20pt}
\textbf{Comment \#10:}
\begin{adjustwidth}{2.5em}{0pt}
\singlespacing \vspace{-10pt}
\textcolor{red}{Page 49, comparison between EM and MaxEnt, how significant is the time saving? Put discussion here. When the network size becomes large and real-time estimation is required, the running time is more important.}
\end{adjustwidth}

\vspace{10pt}
\textbf{Response:}
\begin{adjustwidth}{2.5em}{0pt}
I have adding new discussion on the importance of the time saving. The updated discussion from the thesis is as follows.

\vspace{10pt}
\noindent\textcolor{blue}{Table 6.2 shows the convergence time comparison of both the EM and MaxEnt based solutions to power quality estimation. It can be seen that the convergence time of MaxEnt is much faster as compared to that of EM algorithm. The convergence time is exponentially increasing for the EM algorithm with the network size. For a real-time estimation, the convergence time is more important and the EM based solution is not feasible to operate in real-time given its huge convergence delays. We measure the estimation accuracy using Mean Squared Error (MSE), which is a statistical measure quantifying the difference between values implied by an estimator and the true values of the quantity being estimated. The results are shown in Table 6.3. From the results, we can see that both the methods give very close estimations of the power quality transition functions, i.e., the difference between the estimated and corresponding ground truth functions is negligible.}
\end{adjustwidth}






\vspace{50pt}
 \Large
\noindent \textbf{\underline{Response to Dr. Hong-Chuan Yang:}}
 \large
 
 \vspace{10pt}
\textbf{Comment \#1:}
\begin{adjustwidth}{2.5em}{0pt}
\singlespacing \vspace{-10pt}
\textcolor{red}{Is there a reverse transfer function?}
\end{adjustwidth}

\vspace{10pt}
\textbf{Response:}
\begin{adjustwidth}{2.5em}{0pt}
Yes, in order to calculate the impact of a child node on its parent, we use the reverse transfer function. This concept is explained in greater details in the thesis and quoted as follows.

\vspace{10pt}
\noindent\textcolor{blue}{The problem is inherently challenging as the information received from a power meter flows not only the forward direction from the root nodes toward the leaf nodes, but also in reverse or upstream direction toward the root node (utility main) and back to all other nodes in the network.}

\vspace{10pt}
\noindent\textcolor{blue}{
We calculate the influence of a child device on a parent device. Note that the parent may not necessarily be the immediate parent. To calculate the entropy of parent given child using the general formula of conditional entropy, we need to first calculate the conditional transition function $F$.}

\vspace{10pt}
\noindent\textcolor{blue}{We use the concept of posterior probability (the Bayes theorem) to calculate F. This function is simply the product of the reverse transition functions of devices all the way from child to parent. The reverse transition function $f^\prime(d)$ (consist of $p(parent \mid child)$ or $p(X \mid Y)$) is calculated as $p(X \mid Y) = \frac{p(X) p(Y/X)}{ p(Y)}$.  In our case, the function $f^\prime(d)$ of a device $d$ which list $p(x \mid y)$ in the $xth$ row and $yth$ column is calculated as:
\[f^\prime(d) = \left[\begin{array}{c} f_x(\widehat d)\\ f_x(\widehat d)\\ \vdots\\ f_x(\widehat d) \end{array}\right] \otimes \left[f(d)\right]^T \oslash \left[\begin{array}{c} f_x(d)\\ f_x(d)\\ \vdots\\ f_x(d) \end{array}\right]^T,\]
\noindent where $\otimes$ is the component-wise product, $\oslash$ is the component-wise division, and $\widehat d$ is the immediate parent of device $d$.
Finally:
%\vspace{-0.05in}
\[F = f^\prime(d_o) \times f^\prime(\widehat {d_o}) \times \hdots \times f^\prime(\widecheck {d_i}).\]}
\end{adjustwidth}


\vspace{30pt}
\textbf{Comment \#2:}
\begin{adjustwidth}{2.5em}{0pt}
\singlespacing \vspace{-10pt}
\textcolor{red}{Entropy maximization: Page 16. Put constraints in the problem formulation, change the notation from $R^n$ to $[0,1]^n$}
\end{adjustwidth}

\vspace{10pt}
\textbf{Response:}
\begin{adjustwidth}{2.5em}{0pt}
I have changed the notation from $\vec{x}\in \mathbb{R}^n$ to $\vec{x}\in [0, 1]^n$. Updated text is as follows.

\vspace{10pt}
\noindent\textcolor{blue}{where $\vec{x}\in [0, 1]^n$ is the optimization variable, $A \in \mathbb{R}^{m \times n}$, and $B \in \mathbb{R}^{m \times n}$ are problem parameters;  and \textbf{1} is a vector with all 1's.}
\end{adjustwidth}


\vspace{30pt}
\textbf{Comment \#3:}
\begin{adjustwidth}{2.5em}{0pt}
\singlespacing \vspace{-10pt}
\textcolor{red}{Does the transfer function change over time? }
\end{adjustwidth}

\vspace{10pt}
\textbf{Response:}
\begin{adjustwidth}{2.5em}{0pt}
In Chapter 3, using a real-time PQ dataset, we learned the device latent feature (transfer function). The $k$-fold validation proved that the transfer functions were consistent over time. The same is explained in Chapter 3 and quoted below.

\vspace{10pt}
\noindent\textcolor{blue}{In this chapter, we proposed a device latent feature model which learns a device transfer function from real data. The device transfer function is needed to estimate the power quality values on unmonitored links in the power gird. In order to validate the proposed model, We used a real power quality dataset collected by Schneider Electric Inc. in a power grid in Canada. We demonstrated that the historical data can be used to capture the latent features of a device. The $k$-fold cross-validation technique was used to measure the accuracy of latent features we obtained using our dataset. \textbf{Experimental evaluations showed that the captured latent features are consistent}. The latent features learnt in this chapter are used by our meter placement algorithms proposed in Chapter 5.}
\end{adjustwidth}


\vspace{30pt}
\textbf{Comment \#4:}
\begin{adjustwidth}{2.5em}{0pt}
\singlespacing \vspace{-10pt}
\textcolor{red}{Presentation of thesis. Table of Nomenclature. Move ahead because this nomenclature has been used in all chapters.}
\end{adjustwidth}

\vspace{10pt}
\textbf{Response:}
\begin{adjustwidth}{2.5em}{0pt}
I moved the nomenclature before Chapter 1 (after the Table of contents).
\end{adjustwidth}

\vspace{30pt}
\textbf{Comment \#5:}
\begin{adjustwidth}{2.5em}{0pt}
\singlespacing \vspace{-10pt}
\textcolor{red}{Abstract: (add one more sentence regarding the performance).}
\end{adjustwidth}

\vspace{10pt}
\textbf{Response:}
\begin{adjustwidth}{2.5em}{0pt}
Abstract has been updated for the explanation suggested. The updated lines are as follows.

\vspace{10pt}
\noindent\textcolor{blue}{After deploying power quality meters on selected links, a maximum entropy (MaxEnt) based approach is presented to estimate the power quality on the unmonitored lines. Compared to other existing methods such as Monte Carlo Expectation Maximization (MCEM), the MaxEnt based approach is much faster while maintaining similar estimation accuracy. Convergence time of the MaxEnt algorithm is particularly important when the network size increases and we need to do the estimation in real-time.}
\end{adjustwidth}





\vspace{50pt} 
  \Large
\noindent \textbf{\underline{Response to Dr. Dimitri Marinakis:}}
 \large
 
\vspace{10pt}
\textbf{Comment \#1:}
\begin{adjustwidth}{2.5em}{0pt}
\singlespacing \vspace{-10pt}
\textcolor{red}{Page 24: Table 3.2, Table 3.5, Table 3.6, consistent problem. More explanation.}
\end{adjustwidth}

\vspace{10pt}
\textbf{Response:}
\begin{adjustwidth}{2.5em}{0pt}
I agree with the fact that the frequency of the PQ class $c_{14}$ for all meters is reported as 0 (in Table 3.2) while the same for a single meter (device $d_8$ in Table 3.6) is reported as $2122$. The reason is as follows.

\vspace{5pt}\noindent The power quality classes from $c_1$ to $c_{13}$ represent the power disturbances (described in Table 3.4) while $c_{14}$ represents the nominal power quality (a good/normal power quality). The power quality meters in our data collection network were configured to report only bad power quality events and hence the frequency of PQ class $c_{14}$) is 0 in table 3.2. We also noticed that, in some cases, there are bad power quality events reported at some links while nothing reported by other meters at that time instance. This happens when a device, for instance a UPS, maps a bad quality to good quality. In such cases, we assume a nominal PQ value (PQ class $c_{14}$) at the monitored but unreported points.

\vspace{5pt}\noindent The same explanation is now added in Chapter 3. The revised explanation from the thesis is as follows.

\vspace{5pt}
\noindent\textcolor{blue}{Using IEEE Standard 1159 [2], we classify the power quality events based on the fluctuation of the voltage for a predefined period. There are $14$ different power quality classes defined in the standard, denoted from $c_1$ to $c_{14}$, respectively. Table 3.5 shows samples of the events we classify using the IEEE standard where the power quality class is shown in the last column of the table. The frequency of events belonging to the IEEE power quality class ($c_1$ to $c_{14}$) is shown in Table 3.2. Description of the IEEE power quality classes is provided in Table 3.4.}

\vspace{5pt}
\noindent\textcolor{blue}{The power quality meters in our data collection network were configured to report only bad power quality events. Therefore, the frequency of the nominal voltage events (PQ class $c_{14}$) in Table 3.2 is 0. We noticed that, in some cases, there are bad power quality events reported at some links while nothing reported by other meters at that time instance. This happens when a device, for instance a UPS, maps a bad quality to good quality. In such cases, we assume a nominal PQ value (PQ class $c_{14}$) at the monitored but unreported points.
}
\end{adjustwidth}


\vspace{30pt}
\textbf{Comment \#2:}
\begin{adjustwidth}{2.5em}{0pt}
\singlespacing \vspace{-10pt}
\textcolor{red}{MaxEnt: In which situation MaxEnt will suffer? Does Network topology have impact on this?}
\end{adjustwidth}

\vspace{10pt}
\textbf{Response:}
\begin{adjustwidth}{2.5em}{0pt}
In MaxEnt solution, we divide the power network in sub-networks based on meter positions. Since the input and output links of each subnet are monitored, the network topology outside the subnet has no impact on the MaxEnt solution. For larger subnets, the size of unknown variables increases exponentially. Hence, for subnets of larger sizes, the MaxEnt will take longer to converge. We have discussed this potential problem and a possible solution in our thesis.

\vspace{10pt}
\noindent\textcolor{blue}{The proposed solution may not converge efficiently for sub-nets of larger sizes. The scalability problem arises as the number of unknown variables increases exponentially with the increase in the number of components in a subnet. The idea for the extended work is to divide the larger subnets in logical components where each logical component will represent several physical components. Instead of computing individual transfer functions for each physical device, transfer functions of the logical components could be estimated first. In the second round, transfer functions of individual physical components could be estimated from the logical transfer functions. Note that this is a high level guideline which only serves as a starting point towards the final solution. The final solution could be based on a comprehensive mathematical model. Finally, for the evaluation of such a system, various larger IEEE standard test networks could be used.}
\end{adjustwidth}


\vspace{30pt}
\textbf{Comment \#3:}
\begin{adjustwidth}{2.5em}{0pt}
\singlespacing \vspace{-10pt}
\textcolor{red}{Could you extend the model to capture the temporal feature of transition function? For the four-your data, have you observed any changes of transfer function over time?}
\end{adjustwidth}

\vspace{10pt}
\textbf{Response:}
\begin{adjustwidth}{2.5em}{0pt}
In Chapter 3, using a real-time PQ dataset, we learned the device latent feature (transfer function). The $k$-fold validation proved that the transfer functions were consistent over time. The same is explained in Chapter 3 and quoted below.

\vspace{10pt}
\noindent\textcolor{blue}{In this chapter, we proposed a device latent feature model which learns a device transfer function from real data. The device transfer function is needed to estimate the power quality values on unmonitored links in the power gird. In order to validate the proposed model, We used a real power quality dataset collected by Schneider Electric Inc. in a power grid in Canada. We demonstrated that the historical data can be used to capture the latent features of a device. The $k$-fold cross-validation technique was used to measure the accuracy of latent features we obtained using our dataset. \textbf{Experimental evaluations showed that the captured latent features are consistent}. The latent features learnt in this chapter are used by our meter placement algorithms proposed in Chapter 5.}
\end{adjustwidth}



\vspace{30pt}
\textbf{Comment \#4:}
\begin{adjustwidth}{2.5em}{0pt}
\singlespacing \vspace{-10pt}
\textcolor{red}{Section 6.4.3: How does the number of samples impact the accuracy of CE?}
\end{adjustwidth}

\vspace{10pt}
\textbf{Response:}
\begin{adjustwidth}{2.5em}{0pt}
addressed in comment \#6 below.
\end{adjustwidth}


\vspace{30pt}
\textbf{Comment \#5:}
\begin{adjustwidth}{2.5em}{0pt}
\singlespacing \vspace{-10pt}
\textcolor{red}{Put discussion on the condition when MaxEnt suffers (related to question 2).}
\end{adjustwidth}

\vspace{10pt}
\textbf{Response:}
\begin{adjustwidth}{2.5em}{0pt}
Addressed in comment \#2 above.
\end{adjustwidth}


\vspace{30pt}
\textbf{Comment \#6:}
\begin{adjustwidth}{2.5em}{0pt}
\singlespacing \vspace{-10pt}
\textcolor{red}{Page 65, Page 67 (What is the difference between BP benchmark vs. BP algorithm? How many samples? How does the size of samples impact the results? Add discussion to help readers understand better.}
\end{adjustwidth}

\vspace{10pt}
\textbf{Response:}
\begin{adjustwidth}{2.5em}{0pt}
The BP benchmark is essentially a BP approach/algorithm. We used the BP with large sample size as a benchmark. The sample size for the benchmark was 1 million events on each link. This sample set is larger enough to be used as a benchmark. The BP algorithm was then used to place our meters using a sample size of 10,000 events per link. Experimental results show that the error rate was under 5\% for the BP meter placement solution. We add new discussion in the thesis to clarify this point.

\vspace{10pt}
\noindent\textcolor{blue}{The meter placements are then passed to the belief propagation benchmark to compare the accuracy of the two algorithms in terms of mean-square error (MSE), i.e., the mean error between the estimated and actual transition functions on unmonitored links. We collect a set of 1 million known samples for a given meter configuration and infer the maximum likelihood power quality event that would appear at each unmetered node using belief propagation. We then estimate the error rate for each node. If the inferred event differs from the event given by the MC sample, we add $1/N$ for that sample. The sample size of 1 million is large enough to be used as a benchmark. The mean error rate across all nodes is taken as the final performance metric. As shown in Table 5.2, the MSEs are very small for both algorithms in all networks we tested. The BP algorithm gave slightly better estimations than MinEntropy in some cases at a cost of longer running times.}
\end{adjustwidth}


\vspace{30pt}
\textbf{Comment \#7:}
\begin{adjustwidth}{2.5em}{0pt}
\singlespacing \vspace{-10pt}
\textcolor{red}{Page 45, third constraint. If we remove the constraint, what is the impact?}
\end{adjustwidth}

\vspace{10pt}
\textbf{Response:}
\begin{adjustwidth}{2.5em}{0pt}
The third constraint of the MaxEnt objective function is: the estimated functions $f(d_j)$ satisfy the condition $\prod_j f(d_j) = f(s)$. Since we know the transfer $f(s)$ of the subnet, we should use this known information to accurately estimate the individual transfer functions. Without using this constraint, we can still estimate the individual transfer functions considering the other 3 constraints but the accuracy of the estimated functions may not always be accurate.
\end{adjustwidth} 
 
 
 
 
 
\vspace{50pt}
  \Large
\noindent \textbf{\underline{Response to Dr. Kui Wu:}}
 \large

\vspace{10pt}
\textbf{Comment \#1:}
\begin{adjustwidth}{2.5em}{0pt}
\singlespacing \vspace{-10pt}
\textcolor{red}{Chapter 7: Section 7.3.1, Correlation function. What is the underlying assumption in order to use the correlation function defined here?}
\end{adjustwidth}

\vspace{10pt}
\textbf{Response:}
\begin{adjustwidth}{2.5em}{0pt}
The assumptions to use the Pearson correlation measure include: related pairs, absence of outliers, normality of variables, and linearity. These assumptions may not always be true and hence we cannot use this measure as a complete solution. I have updated this explanation in the revised version of the thesis.

\vspace{10pt}
\noindent\textcolor{blue}{Although this technique is simple to use, there are certain assumptions to be true in order to effectively use it. These assumptions include: related pairs, absence of outliers, normality of variables, and linearity. These assumptions may not always be true and hence we cannot use this measure as a complete solution.}

\vspace{10pt}
\noindent\textcolor{blue}{We demonstrate various scenarios where the above correlation assumptions do not hold and hence we cannot effectively use it to accurately detect a malfunction device in the power network. We classify the identified scenarios in two classes: 1) a malfunction device is not detected; and 2) a better PQ producing device is classified as a malfunction.}
\end{adjustwidth}


\vspace{30pt}
\textbf{Comment \#2:}
\begin{adjustwidth}{2.5em}{0pt}
\singlespacing \vspace{-10pt}
\textcolor{red}{Suggestion: Title (reconsider)}
\end{adjustwidth}

\vspace{5pt}
\textbf{Response:}
\begin{adjustwidth}{2.5em}{0pt}
I have changed the title to:

\vspace{5pt}
\noindent\textcolor{blue}{An Analytical Framework for Power Quality Monitoring in Enterprise Level Power Grid}
\end{adjustwidth}


\vspace{30pt}
\textbf{Comment \#3:}
\begin{adjustwidth}{2.5em}{0pt}
\singlespacing \vspace{-10pt}
\textcolor{red}{Equation 7.1 --$>$ Equation (7.1). Double check the whole thesis for the missing ( ).}
\end{adjustwidth}

\vspace{10pt}
\textbf{Response:}
\begin{adjustwidth}{2.5em}{0pt}
I have changed the Equation 7.1 to Eq. (7.1) to make it consistent with other equation references in the thesis.
\end{adjustwidth}

\end{document}